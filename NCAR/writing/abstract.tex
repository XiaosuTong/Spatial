\section{Spatial and Temporal Modeling with LOESS and D\&R}

\subsection{LOESS}
Local regression methods model the relationship between an independent and dependent variable 
through weighted fitting of polynomials in local neighborhoods of the design space. A popular 
method, loess, is a local regression method with favorable statistical and computational properties.
On the one hand, Loess modeling has been adapted to the modeling of time series data with 
deterministic seasonal and trend components with the STL method (seasonal trend decomposition using 
loess). On the other hand, Loess modeling can be easily applied to spatial data in two dimensions. 
It is nothing but a quadratic fit with respect to longitude and latitude. 

\subsection{STL+}
Several improvements have been made to the STL implementation, packaged as “stl2”. The main function
of the package is stl2(), which has the same arguments as the original stl() function. In addition, 
it offers the following. First, local quadratic support for seasonal, trend, and low-pass loess 
smoothing by specifying s.degree = 2, etc. Second, loess cubic interpolation is used instead of 
linear interpolation, using the estimated derivatives. Third, blending for seasonal, trend, and 
low-pass loess smoothing by specifying $\delta$, e.g. s.blend=0.5, etc. Forth, further trend 
smoothing by specifying vectors fc.window and fc.degree for the spans and degrees of the post-trend 
components to be estimated. These are carried out in the order they are specified. Fifth, capability
to handle missing value.

\subsection{Divide and Recombined (D\&R)}
Divide and recombined concept has been applied to the analysis through various place. 

\subsection{Climate Data}
The data set we are going to download is about observed monthly total precipitation and monthly 
average minimum and maximum daily temperatures for the coterminous US 1895-1997. Totally, there are 
12,392 stations all over the nation, 8,125 stations for temperature, 11,918 stations for 
precipitation. For each station, an unique ID, station name, elevation, longitude, and altitude are
available. If a measurement of a specific station at a specific month is treated as one observation,
then there are 6,204,442 observations for precipitation and 4,285,841 observations for temperature. 

Third paragraph talks something about the modeling and diagnostics.
However,missing value is one of the problem for this climate data. In order to examinate the 
temporal modeling using STL+, 100 stations will no missing values are selected randomly from 
stations. 

Forth paragraph talks something about tuning parameter selection.
In the experiment 1, we are trying to use 600 observations to predict oncoming 36 months of 
maximum temperature. Then the predicton error is calculated to measure the prediction ability. 
We want to see how the different parameter setting affect the prediction ability. For each station,
we use the first 600 observations to predict 36 oncoming observations, then move the range window 
of training dataset one observation ahead, and predict the next 36 observations. There are 1236 
observations for each station, so we conduct 601 replicates for each station.

