\section{Spatial and Temporal Modeling}

\subsection{LOESS for Spatial}
Local regression methods model the relationship between an independent and dependent variable 
through weighted fitting of polynomials in local neighborhoods of the design space. A popular 
method, loess, is a local regression method with favorable statistical and computational properties.
On the one hand, Loess modeling has been adapted to the modeling of time series data with 
deterministic seasonal and trend components with the STL method (seasonal trend decomposition using 
loess). On the other hand, Loess modeling can be easily applied to spatial data in two dimensions. 
It is nothing but a quadratic fit with respect to longitude and latitude. 

\subsection{STL+ for Temporal}
STL is a filtering procedure for decomposing a seasonal time series into three components: trend,
seasonal, and remainder. The long term change in the time series is captured by the trend component.
A cyclical pattern is reflected in the seasonal component. The residuals, the remaining variation, 
are the remainder component.
Several improvements have been made to the STL implementation, packaged as “stl2”. The main function
of the package is stl2(), which has the same arguments as the original stl() function. In addition, 
it offers the following. First, local quadratic support for seasonal, trend, and low-pass loess 
smoothing by specifying s.degree = 2, etc. Second, loess cubic interpolation is used instead of 
linear interpolation, using the estimated derivatives. Third, blending for seasonal, trend, and 
low-pass loess smoothing by specifying s.blend=0.5, etc. Forth, further trend smoothing by 
specifying vectors fc.window and fc.degree for the spans and degrees of the post-trend components to 
be estimated. These are carried out in the order they are specified. Fifth, capability to handle 
missing value.

\subsection{Climate Data}
The data set we are going to download is about observed monthly total precipitation and monthly 
average minimum and maximum daily temperatures for the coterminous US 1895-1997. Totally, there are 
12,392 stations all over the nation, 8,125 stations for temperature, 11,918 stations for 
precipitation. For each station, an unique ID, station name, elevation, longitude, and altitude are
available. If a measurement of a specific station at a specific month is treated as one observation,
then there are 6,204,442 observations for precipitation and 4,285,841 observations for temperature. 

However, missing value is one of the problem for this climate data. Less than 10 percent of stations
have no missing observations across the 1,236 months. Less than 20 percent of stations only have 800
valid observations during the 103 years. Measurement status of monthly observation for each station
has been visualized. It is reasonable to assume that all missing values happened at random time 
points. In order to examinate the temporal modeling using STL+, 100 stations will no missing values 
are selected randomly. 

Tuning parameter selection is one of the most important step for all nonparametric methods. Multiple 
experiments with different smoothing parameters were performed in this work, which used 600 
observations to predict oncoming 36 months of observations. Then the range window of training 
dataset one was moved observation ahead, and the next 36 observations were predicted. The predicton 
error is calculated to measure the prediction ability. For each of the 100 stations, normal quantile
plot of the prediction error conditional on the prediction lag 

For each station, we use the first 600 
observations to predict 36 oncoming observations,  There are 1236 observations for each 
station, so we conduct 601 replicates for each station. Divide and recombined computation concept 
has been applied here to parallelly 
